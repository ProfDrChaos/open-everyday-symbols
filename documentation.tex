\documentclass{article}

\usepackage{open-everyday-symbols}
\usepackage[margin=.95in]{geometry}

\begin{document}

\title{The \emph{OPEN} Everyday Symbols Library\\[1ex]\large Version 1.0}
\author{Pascal Bercher\\pascal.bercher@anu.edu.au}
\date{\today}

\maketitle

This package is an \emph{open} and continuously extended list of ``everyday symbols'' (i.e., not of a mathematical or otherwise primarily technical nature). I call it open because I ask \emph{everybody} (i.e., also potentially you) to contribute your own symbols to it. I hope that enough people consider creating their own symbols, e.g., with the help of chatGPT, and contribute them to this package. (Note that creating such graphics isn't particularly hard or time-intense when getting assisted by generative AI such as chatGPT. It will take you a bit longer to create variants of the single symbol you need and to contribute your work to this package, but if enough people do this, we'll get a decent package at some point.)

\medskip
Every command follows the structure:
\begin{center}
  \texttt{\string\everydaySymbol[some-color]$\{$main-category$\}\{$sub-identifier-number$\}$}
\end{center}
Here, \texttt{some-color} is optional, default is black/white (all standard \LaTeX{} colors are possible). Some graphics allow further arguments for configuration (in which case ``some-color'' has to be replaced by ``color=some-color''). If a symbol is too large (or small), simply scale it up or down using
\texttt{\string\scalebox$\{$factor$\}\{$your-symbol$\}$}.

\medskip
Note that the document is ordered by category, like doors, houses, etc. Note however that every main category can have multiple sub categories (the sub-identifier), where one sub category is by one author, meaning that for each main category you might find several further interpretations and variants of it. 

\medskip
The rule for the sub-categories' name is ``descriptor--author--ID''. The ID is to enumerate over different symbols, whereas author allows multiple people to contribute their work without incurring a name clash.

\subsection*{More symbols required?}

You might be here because you require some (non-mathematical) symbol... Thus, two pieces of advice:

\begin{itemize}
  \item Check out https://detexify.kirelabs.org/classify.html: it allows you to draw a symbol in the browser, and text recognition recommends \LaTeX{} commands and the packages in which they are found.
  \item Other than that, one might only need this amazing PDF:\\
  https://tug.ctan.org/info/symbols/comprehensive/symbols-a4.pdf. 
  It lists out dozens (or hundreds?) of packages and their symbols, including their commands and symbols. Thus, either scroll through its almost 500 pages, or use full-text search, e.g., for ``dinosaur''.  
\end{itemize}

If you still don't find what you are after, please consider drawing variants of what you need yourself using chatGPT in tikz, and ``donate'' your work into this package so that it grows. :)

Version history at the very end.

\pagebreak

\section{Doors}

In this section, you find multiple variants of doors.


\subsection{floorplan--pascal}

Author: package author (Pascal; using chatGPT).

\begin{itemize}
  \foreach \i in {01,02,03,04} {
    \item \texttt{\string\everydaySymbol\{door\}\{simple--pascal-\i\}} \dotfill\ 
          \everydaySymbol{door}{simple--pascal-\i}
  }

    % item 5 is shown separately as otherwise, strangely, the symbols are not perfectly aligned
    \item \texttt{\string\everydaySymbol\{door\}\{simple--pascal-05\}} \dotfill\ 
          \everydaySymbol{door}{simple--pascal-05}

    \item \texttt{\string\everydaySymbol[color=red!66!yellow]\{door\}\{simple--pascal-05\}} \dotfill\ 
          \everydaySymbol[color=red!66!yellow]{door}{simple--pascal-05}

    \item \texttt{\string\everydaySymbol[blue]\{door\}\{simple--pascal-05\}} \dotfill\ 
          \everydaySymbol[blue]{door}{simple--pascal-05}
\end{itemize}




\section{Houses}

In this section, you find multiple variants of houses.

\subsection{iconic--pascal}

Author: package author (Pascal; using chatGPT). Note that these commands require keys to specify the exact graphic. (The code of this example might hence also be interesting to developers for further symbols.)

\begin{itemize}
  \item \texttt{\string\everydaySymbol\{house\}\{simple--pascal\}} \dotfill\ 
        \everydaySymbol{house}{simple--pascal}

  \item \texttt{\string\everydaySymbol[config=\{middleDoor,chimney\}]\{house\}\{simple--pascal\}} \dotfill\ 
        \everydaySymbol[config={middleDoor,chimney}]{house}{simple--pascal}

  \item \texttt{\string\everydaySymbol[config=\{rightDoor,leftWindow\}]\{house\}\{simple--pascal\}} \dotfill\ 
        \everydaySymbol[config={rightDoor,leftWindow}]{house}{simple--pascal}

  \item \texttt{\string\everydaySymbol[config=twoWindows,color=red]\{house\}\{simple--pascal\}} \dotfill\ 
        \everydaySymbol[config=twoWindows,color=red]{house}{simple--pascal}

  \item \texttt{\string\everydaySymbol[config=\{twoWindows,middleDoor\}]\{house\}\{simple--pascal\}} \dotfill\ 
        \everydaySymbol[config={twoWindows,middleDoor}]{house}{simple--pascal}

\end{itemize}

\section*{Version History}

\begin{itemize}
  \item Version 1.0: First version with just two categories: Door (simple--pascal) and House (simple--pascal)
\end{itemize}




\end{document}